\section{Leakage on Hardware Information}

%Different processors result into different response time to the same message; hence the response time can be correlated to the MCU.

The nature of sensor networks suggests that hardware information might became valuable in terms of security. For example, leakage of hardware information could help the adversary to identify devices with specific vulnerabilities, or provides hints of target's functionality which relates to the encrypted communication.

Although network protocols are generally implemented independently to hardware, information could still leak through various ways. For instance, when the default MAC address is used, one can immediately look up the manufacturer by the Burned-In Address (BIA)\cite{BIA}.

Different driver implementations may also cause leakage. During the experiments, we realised that TelosB\cite{TelosB} discards all packets exceeding 127 bytes\footnote{MTU specified by 802.15.4 standard.}; whereas CC2538 is capable of handling packets even up to 160 bytes. Therefore an adversary can immediately rule out TelosB when any packet larger than 127 bytes is observed.

%Now it comes to PING.
A more general leakage is the response latency to a specific message, as different processors would have different computational power and thus different time to process a same message. The ICMP messages turns out to be suitable for this attack since they are standardised and thus universally supported. Among the ICMP messages, PING is especially ideal for two reasons: 
\begin{enumerate}
	\item It is mandatory in the ICMP standard.
	\item It only swaps the source and destination address of the packet; thus minimises different code path in protocol processing.
\end{enumerate}

\begin{table}{r}{0.5\textwidth}
	\center
	\begin{tabular}{|c|c|c|}
	\hline
			& CC2538	& TelosB \\ 
	\hline
	Average(ms)	& 9.56		& 17.03 \\ 
	\hline
	Range(ms)	& [9.16, 10.06]	& [16.49, 17.68]	\\
	\hline
\end{tabular}

	\caption{PING Response Latency\label{PingResponse}}
\end{table}

\Cref{PingResponse} shows the PING response latency on CC2538 and TelosB. The result confirms that these devices can be distinguished by PING response latency.

%\begin{itemize}
%	\item PING is a ideal feature in ICMP to launch this attack.
%\end{itemize}

%The processor can be considered a hint for the adversary to find out the model of target device.
