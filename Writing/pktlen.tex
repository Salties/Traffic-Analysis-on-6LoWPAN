\section{Exploiting Packet Length Information\label{sec: PacketLen}}

As our brief survey of traffic analysis via exploiting packet lengths showed in \Cref{subsec: RelatedWork}, the packet length has proven to be a powerful side channel for the classical Internet protocols. It is worth noting that this side channel is `noisy' in the classical Internet setting: websites or web applications in this setting typically feature advertisements, which impact on packet lengths; TCP/IP allows to fragment packets and then reassembles them, a feature which is not presented in UDP. Thus, due to the nature of UDP exploiting the packet length as side channel should be easier in the IoT setting.

Clearly then, any web application style implementations involving an IoT device will thus be extremely vulnerable to attacks such as \cite{WebSidechannel}. In the absence of this scenario for state-of-the art IoT applications, it still sends a cautionary warning to developers: binary responses (e.g. `yes' vs. `no', or `on' vs. `off') must always be coded via a binary variable and not via strings because these will have different lengths, which are directly visible via the packet length.

In the remainder of this section we will highlight further problems that arise if packet lengths leak information.

\subsection{Distinguishing ICMP Messages}
The Internet Control Message Protocol(ICMP)\cite{rfc4443} performs the management tasks in a network, such as link establishment and routing information exchange. As explained before we utilise the open source system Contiki, which supports a (sub)set of the ICMP standard (we list the supported ICMP messages further below). Many ICMP messages are ideal for network discovery and exploration, although the purpose of ICMP is to send error messages to the source IP address if standard IP packets fail to be transmitted correctly. 


\begin{itemize}
	\item \textbf{DAG Information Object (DIO)} \\
	DIO contains the 6LoWPAN global information. It could be periodically broadcasted for network maintenance, or unicasted to a new joining node as a reply to DIS (see below).
	\item \textbf{DAG Information Solicitation (DIS)} \\
	DIS is sent by a newly started node to probe any existing 6LoWPANs. A DIO would be replied if the DIS is received by any neighbour nodes.
	\item \textbf{Destination Advertisement Object (DAO)} \\
	DAO is sent by a child node to its precedents (The 6LoWPAN DODAG topology is defined in \cite{rfc6550}) to propagate its routing information.
	\item \textbf{Neighbour Solicitation (NS) and Neighbour Advertisement (NA)} \\
	NS and NA are the ARP replacement in IPv6, where NS queries a translation and NA answers one. In addition, they are also used for local link validity check.
	\item \textbf{Echo Request and Echo Response (PING)} \\
	Echo Request and Echo Response are well known as the PING packets. They are mostly used for diagnostic purposes, such as connectivity test or Round Trip Time (RTT) estimation. Echo Request may contain arbitrary user defined data and Echo Response simply echoes its corresponding request.
\end{itemize}

Generally, ICMP messages can be protected by either using the secure ICMP messages as described in \cite{rfc4443}, or relying on the lower layer encryption provided by 802.15.4. Contiki OS does not have the former implemented, hence 802.15.4 security is the only option currently. We simulated a 6LoWPAN network with 802.15.4 security enabled (with strongest encryption and authentication). We configured the nodes to also generate random UDP packets. Despite the fact that all ICMP messages were encrypted, our experiments show that several ICMP messages can be identified by their packet size and MAC destination. \Cref{IcmpPacketFeature} summarises the packet features. The value $x$ denotes the size of user defined data in bytes.

\begin{table}
	\center
	{
	\begin{tabular}{|c|c|c|}
		\hline
		       & Packet Size (bytes) & Type of MAC Destination \\ \hline
		DIS    & 85                  & broadcast                       \\ \hline
		DIO  & 118/123                 & broadcast/unicast                       \\ \hline
		DAO    & 97                  & unicast                      \\ \hline
		NS & 87                  & broadcast/unicast                       \\ \hline
		NA     & 87                  & unicast                      \\ \hline
		PING   & $101+x$               & unicast                      \\ \hline
		UDP Multicast   & $85+x$                  & broadcast               \\ \hline
		UDP Unicast   & $107+x$                  & unicast                       \\ \hline
	\end{tabular}
}

	\caption{6LoWPAN Packet Features\label{IcmpPacketFeature}}
\end{table}

%First of all, a DIS has the unique smallest packet size of 85 bytes, indicating it can be easily identified.

Among the unicast packets, PING and UDP have at least 101 and 108 bytes\footnote{PING can be sent without user defined data and UDP packets requires at least 1 byte.}. Therefore, DAO can be uniquely identified as the shorter unicast packet of $97$ bytes.  For the same reason NA and unicast NS can also be distinguished from other packets by filtering packets of $87$ bytes. Considering that NA is sent as a response to NS according to the protocol, one can always identify the first being NS and second being NA. 

Similarly, unicast DIO can be identified as the 123 bytes packet followed by DIS, where the later has a unique 85 byte size. However, there is a potential of false positive induced by PING or UDP packets with user defined data crafted to have the same packet length\footnote{22 bytes for PING and 16 bytes for UDP.}. PING could be recognised by its pair-wised appearance, as the response would have nearly the same meta data as the original request, except the exchanged source and destination. For broadcast packets, DIS can be easily identified by its unique 85 bytes packet size. Others like broadcast NS can be identified by the followed characteristic NA response; and packets of 118 bytes those are periodically broadcasted are likely to be DIOs.

In summary, among all the packets, DAO, NA, NS, DIS can be identified with certainty. DIO and PING cannot be certainly identified but they both have significant characters. Notice that the above contained all ICMPv6 messages supported by Contiki; therefore UDP packets can be reversely filtered, although in some cases they get mixed with DIO and PING.

Although leakage in ICMP messages does not directly lead to any breach of application data, it would still be harmful by providing the adversary with information about the state of the network, including which nodes recently joined etc. Specifically DAO is always sent from a child to its parent and can be uniquely identified; therefore together with MAC addresses the adversary may exploit it to draw a graph that shows the parental relations in the network. In addition, these information can also be exploited by attacks as in \cite{WsnIcmp}.

%\textbf{
%TODO: why would you get the network topology (do you mean via observing the different MAC addresses alongside what they send out)? AFter all the payload is still encrypted and so you don't see the actual DAO for instance??
%}
\subsection{Distinguishing Different Devices}
%Gain hardware information.
%\begin{itemize}
%	\item Some devices can handle longer packets, some can't.
%	\item Works on both 802.15.4 Security and DTLS.
%\end{itemize}

In the classical Internet world, ICMP has been well known for its use for OS fingerprinting\cite{OsFingerprint}. In the case of the IoT, this could be possible as well (as different OS support different subsets of ICMP), however an additional attack vector exists. This is because different IoT devices have different hardware limitations or drivers. We noticed that our TelosB\cite{TelosB} discards all packets exceeding 127 bytes\footnote{MTU specified by 802.15.4 standard.} whereas our CC2538 handles packets even up to 160 bytes. Therefore an adversary can immediately rule out TelosB whenever a packet larger than 127 bytes processed by the target.
