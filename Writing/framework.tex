\documentclass{article}

%Common packages.
\usepackage{cite}
\usepackage{amsmath}
\usepackage{amssymb}
\usepackage{graphicx}
\usepackage{listings}
\usepackage{url}
\usepackage{appendix}
\usepackage{adjustbox}
\usepackage{subcaption}
\usepackage{amsthm}
\usepackage{ltxtable}
\usepackage{rotating}

%NOT TO BE INCLUDED IN SUBMISSION.
\usepackage{hyperref}

%Need to be at last.
\usepackage{cleveref}

\newcommand{\AddFigure}[3]
{
	\begin{figure}[ht!]
		\center
		\includegraphics[width=\linewidth]{#1}
		\caption{#2}
		\label{#3}
	\end{figure}
}

\newcommand{\AddCode}[3]
{
	\lstinputlisting[captionpos=b, caption=#2, breaklines=true,label=#3]{#1}
}

\theoremstyle{definition}
\newtheorem{example}{Example}


\title{Traffic Analysis On 6LoWPAN}
\author{Yan Yan}
\date{\today}

\begin{document}

\maketitle

%\tableofcontents
\section{Introduction}

A typical sensor network is built as described in \Cref{Protocols}.

\begin{table}[!h]
	\centering
	\begin{tabular}{|c|c|}
\hline
Physic                        & \multirow{2}{*}{802.15.4} \\ \cline{1-1}
Link                          &                           \\ \hline
Network                       & 6LoWPAN                   \\ \hline
\multirow{2}{*}{Transmission} & UDP                       \\ \cline{2-2} 
                              & DTLS*                     \\ \hline
Application                   & CoAP / CoAPs*             \\ \hline
\end{tabular}

	\caption{Protocol stack for sensor networks. (* are optional.)\label{Protocols}}
\end{table}


\begin{itemize}
	\item Security options in 6LoWPAN: 802.15.4 Security and DTLS.
	\item Eavesdropping is easier for 6LoWPAN.
\end{itemize}

We discussed packet features independently in this paper but in practice they can be analysed jointly to reveal more leakage.

\section{Packet Length}
Packet length is proven to be a major leakage source in Internet traffic analysis attacks. Since neither 802.15.4 Security nor DTLS protects packet length, we expect the same attacks would apply to 6LoWPAN traffic as well. Further more, we suppose 6LoWPAN applications are even more vulnerable to the leakage as there tends to be less noise in the traffic.

Unreliable transmission is more preferable for 6LoWPAN applications for its low cost. Take UDP, the symbolic protocol for unreliable transmission, for example. UDP has no flow control; therefore packet size as well as number of packets are deterministic given the same application data. This effectively removed the natural noise induced by TCP packet reassembling. Although packet fragmentation is also provided by IPv6 but this feature is usually disabled for practical reasons.

``Random" data, such as ads, is another major noise in Internet traffic which does not exist in 6LoWPAN traffic. The random data naturally randomises the traffic and hence complicates traffic analysis attacks. The leakage would be more easily observed without these noise.

\subsection{ICMP Leakage}
Plaintext recovery.
\begin{itemize}
	\item The attack combines packet length with MAC address information.
	\item Works on 802.15.4 Security. No need for DTLS (ICMP is plaintext when DTLS.)
\end{itemize}

\subsection{NEW: Hardware Leakage}
Gain hardware information.
\begin{itemize}
	\item Some devices can handle longer packets, some can't.
	\item Works on both 802.15.4 Security and DTLS.
\end{itemize}

\subsection{Countermeasure}
Pad to MTU is recommended as it completely hides the length of plaintext. This method only induces slight overhead as 6LoWPAN MTU is small and requires only negligible computation.

Unifying the drivers by strictly apply the standard MTU. Applications should be compatible to the standard anyway. 

Inserting dummy packets. More overhead to bandwidth and energy.

\section{Response Time}
Response time is another major feature exploited in Internet traffic analysis attack. Again we would expect the same attacks can be applied to 6LoWPAN traffic and works even better than their counterparts against Internet traffic, as the accuracy of timing measurements can be greatly improved for 6LoWPAN traffic. 

Firstly, the devices are physically close to each other and uses RF to communicate. The adversary can remove the RTT noises by measure the packets on the server side. 

Secondly, performance of the constrained devices are low; hence gives a better resolution on timing code execution.


\subsection{Different Sensors}
Plaintext recovery.
\begin{itemize}
	\item Reading different sensors takes different time.
	\item Requirement: needs to identify CoAPS packets.
	\item Works on both 802.15.4 Seuciryt and DTLS, if the requirement is met. 
\end{itemize}

\subsection{Different Hardware}
Gain hardware information.
\begin{itemize}
	\item Different hardware has different response latency against the same message.
	\item For 802.15.4: Requires additional information. Can be satisfied by the previous ICMP attack.
	\item For DTLS: No requirement. The adversary can actively send a message, e.g. PING.
\end{itemize}

\subsection{Running Code Fingerprint}
Fingerprints the code running on a device, which leads to plaintext recovery.
\begin{itemize}
	\item Theory explained in \Cref{FingerprintTheory}.
	\item Requirement: needs to actively send messages. Any Request/Response protocol works, such as PING, DTLS Heartbeat, CoAP PING, etc. 
	\item Not for 802.15.4: adversary can't join the network.
	\item For DTLS: PING is available and confirmed. DTLS Heartbeat should work if supported.
\end{itemize}

\AddFigure{fig/PingProbe_Session.png}{How fingerprint works}{FingerprintTheory}

\subsection{Countermeasure}
Two options:
\begin{itemize}
	\item Randomise response time: needs to be resilient to statistical analysis.
	\item Threashold response time (discard or reply in a constant time): performance tradeoff. Recommended for unreliable transmissions.
\end{itemize}

\section{Conclusion}

\bibliographystyle{IEEEtran}
\bibliography{references,rfc}

%\appendix
%\section{appendix.tex}

\end{document}
