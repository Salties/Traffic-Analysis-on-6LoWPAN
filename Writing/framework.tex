\documentclass{article}

%Common packages.
\usepackage{cite}
\usepackage{amsmath}
\usepackage{amssymb}
\usepackage{graphicx}
\usepackage{listings}
\usepackage{url}
\usepackage{appendix}
\usepackage{adjustbox}
%\usepackage{subcaption}
\usepackage{amsthm}
\usepackage{ltxtable}
\usepackage{rotating}
\usepackage{wrapfig}
\usepackage{multirow}
\usepackage{placeins}

%NOT TO BE INCLUDED IN SUBMISSION.
\usepackage{hyperref}

%Need to be at last.
\usepackage{cleveref}

\newcommand{\AddFigure}[3]
{
	\begin{figure}[ht!]
		\center
		\includegraphics[width=\linewidth]{#1}
		\caption{#2}
		\label{#3}
	\end{figure}
}

\newcommand{\AddCode}[3]
{
	\lstinputlisting[captionpos=b, caption=#2, breaklines=true,label=#3]{#1}
}

\theoremstyle{definition}
\newtheorem{definition}{Definition}
\theoremstyle{example}
\newtheorem{example}{Example}


\title{Traffic Analysis On 6LoWPAN}
\author{Yan Yan}
\date{\today}

\begin{document}

\maketitle

%\tableofcontents
\section{Introduction}

\begin{itemize}
	\item Security options in 6LoWPAN: 802.15.4 Security and DTLS.
\end{itemize}

\section{Packet Length}
\begin{itemize}
	\item TLS Attacks still works and even better against DTLS due to protocol feature.
\end{itemize}

\subsection{ICMP Leakage}
Plaintext recovery.
\begin{itemize}
	\item The attack combines packet length with MAC address information.
	\item Works on 802.15.4 Security. No need for DTLS (ICMP is plaintext when DTLS.)
\end{itemize}

\subsection{NEW: Hardware Leakage}
Gain hardware information.
\begin{itemize}
	\item Some devices can handle longer packets, some can't.
	\item Works on both 802.15.4 Security and DTLS.
\end{itemize}


\section{Response Time}
\begin{itemize}
	\item TLS attacks still works and even better against DTLS for no RTT noise. 
\end{itemize}

\subsection{Different Sensors}
Plaintext recovery.
\begin{itemize}
	\item Reading different sensors takes different time.
	\item Requirement: needs to identify CoAPS packets.
	\item Works on both 802.15.4 Seuciryt and DTLS, if the requirement is met. 
\end{itemize}

\subsection{Different Hardware}
Gain hardware information.
\begin{itemize}
	\item Different hardware has different response latency against the same message.
	\item For 802.15.4: Requires additional information. Can be satisfied by the previous ICMP attack.
	\item For DTLS: No requirement. The adversary can actively send a message, e.g. PING.
\end{itemize}

\subsection{Running Code Fingerprint}
Fingerprints the code running on a device, which leads to plaintext recovery.
\begin{itemize}
	\item Theory explained in \Cref{FingerprintTheory}.
	\item Requirement: needs to actively send messages. Any Request/Response protocol works, such as PING, DTLS Heartbeat, CoAP PING, etc. 
	\item Not for 802.15.4: adversary can't join the network.
	\item For DTLS: PING is available and confirmed. DTLS Heartbeat should work if supported.
\end{itemize}

\AddFigure{fig/PingProbe_Session.png}{How fingerprint works}{FingerprintTheory}

\section{Conclusion}

\bibliographystyle{IEEEtran}
\bibliography{references,rfc}

%\appendix
%\section{appendix.tex}

\end{document}
