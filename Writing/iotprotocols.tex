
\section{A Typical IoT Protocol Stack\label{sec: IotProtocols}} 

%IoT Protocols
There are large number of protocols, which have been proposed for different IoT applications adapting to various requirements. For example, some smart houses simply use WiFi for connectivity and VANETs\footnote{Vehicular ad hoc networks} may adopt DSRC\cite{DSRC}. 

%WSN Protocols
In this paper we focus on 6LoWPAN\cite{rfc4944} which is based on the 802.15.4\cite{802154} standard. These standards are designated for constrained environments such as Wireless Sensor Networks,  but other competing standards exist at different layers. Bluetooth Low Energy(BLE)\cite{BLE} is a strong competitor to 802.15.4 as well as the LiFi\cite{LiFi} technology. Zigbee\cite{Zigbee} was originally intended as a collective protocol over 802.15.4 but it has been recently adapted to IP-based network in ZigbeeIP\cite{ZigbeeIp}. The RIME stack\cite{RIME} proposes a set of non-layered primitives over 802.15.4 but it is likely to be phased-out due to the lack of of interoperability with the TCP/IP protocol stack. 

%6LoWPAN OS
6LoWPAN thus is the most popular standard for low power networks, and thus it is supported by several competing IoT Operating Systems, including Contiki OS\cite{Contiki}, OpenWSN\cite{OpenWSN}, FreeRTOS\cite{FreeRTOS} and the recent RIOT\cite{RIOT}. We chose Contiki OS for our experiments because it is easy to customise.

\subsection{Our experimental network}
%Say something about devices, and choice of OS, etc.
%Hardware
Our experimental network is constructed using two different devices. These are a TelosB\cite{TelosB} and a CC2538\cite{CC2538}. The TelosB is a low cost sensor powered by an MSP430 with an AES co-processor. It represents typical low-end devices. The CC2538 is the high end device powered by an ARM Cortex-M3 with multiple cryptographic processors including AES, RSA, SHA-2 and ECC, suggesting that it is suitable to develop secure applications. %Should I cite our previous paper...?

%Software
Both devices are supported by the Contiki OS. We adopted the default settings of the Contiki OS, except for enabling 802.15.4 security\cite{802154} for some experiments. Note that the Contiki MAC\cite{ContikiMAC} is chosen by default over TSCH\cite{TSCH}. For Layer 4\cite{OSI} and above protocols, we went with the widely accepted combination of CoAP\cite{rfc7252}, and DTLS\cite{rfc6347}(optional) over UDP\cite{rfc768}\footnote{CoAPs is equivalent to CoAP over DTLS.}. \Cref{Protocols} summarises our choice of protocol stack.

\begin{table}[!h]
	\centering
	\begin{tabular}{|c|c|}
		\hline
		Physical                        & \multirow{2}{*}{802.15.4} \\ \cline{1-1}
		Link                          &                           \\ \hline
		Network                       & 6LoWPAN                   \\ \hline
		\multirow{2}{*}{Transmission} & UDP                       \\ \cline{2-2} 
		& DTLS*                     \\ \hline
		Application                   & CoAP / CoAPs*             \\ \hline
	\end{tabular}
	
	\caption{Protocol stack for our experiments(* is optinal)\label{Protocols}}
\end{table}

%\begin{itemize}
%	\item Security options in 6LoWPAN: 802.15.4 Security and DTLS.
%	\item Eavesdropping is easier for 6LoWPAN.
%\end{itemize}

%Avaible security options.
\subsubsection{802.15.4 and DTLS}
In our setting, there are two standards available for packet encryption, namely 802.15.4 security\cite{802154} and DTLS\cite{rfc6347}. 802.15.4 security is provided by the noncoresec\cite{noncoresec} API, which implements 802.15.4 authenticated encryption with AES-128 CCM*\cite{CCM} using a hard-coded key shared by the whole 6LoWPAN network. We chose tinyDTLS\cite{tinydtls}  as library for the DTLS protocols, because it provides a minimum DTLS implementation that supports two ciphersuites which are TLS\_PSK\_WITH\_AES\_128\_CCM\_8\cite{rfc6655} and TLS\_ECDHE\_ECDSA\_WITH\_AES\_128\_CCM\_8\cite{rfc6655} respectively. Evidently, they both utilise AES-128 CCM* as the packet encryption method.