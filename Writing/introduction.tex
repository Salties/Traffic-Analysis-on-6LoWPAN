\subsection*{Abstract}
The Internet of Things (IoT) has become a reality: small connected devices feature in everyday objects
including childrens' toys, TVs, fridges, heating control units, etc. Supply chains feature sensors
throughout, and significant investments go into researching next-generation healthcare, 
where sensors monitor wellbeing. A future in which sensors and other (small) devices interact to create sophisticated applications seems just around the corner. All of these applications have a fundamental need for security and privacy and thus cryptography is deployed as part of an attempt to secure them. In this paper we explore a particular type of flaw, namely side channel information, on the protocol level that can exist despite the use of cryptography. Our research investigates the potential for utilising packet length and timing information (both are easily obtained) to extract interesting information from a system. We find that using these side channels we can distinguish between devices, different programs running on the same device including which sensor is accessed. We also find it is possible to distinguish between different types of ICMP messages despite the use of encryption. Based on our findings, we provide a set of recommendations to efficiently mitigate these side channels in the IoT context. 

%\keywords{6LoWPAN, Side Channels, Traffic Analysis}

\section{Introduction}
The expression `Internet of Things' (IoT) can refer to a multitude of objects and protocols, which share that they have been purposefully designed for resource constraint environments. Whereas the typical TCP/IP network stack produces considerable overhead to achieve quality of service for applications that are based on it, the nature of many IoT `things' is such that a full implementation of it would not be practical. Often `things' are sensor, which are devices that have to function on little resources (most importantly power). Thus a whole host of new networking protocols have been developed over the years to cater for such resource constrained devices: 6LoWPAN is the `tiny' version of IPv6, UDP tends to be used instead of TCP/IP, DTLS can be used for end-to-end security or one can directly invoke 802.15.4 security which is part of 6LoWPAN, and finally CoAP(s) is the replacement for HTTP(s). Thus there are two options (802.15.4, and DTLS) to secure communications between the `things' and a server/gateway. 

Implementing cryptography correctly and securely has proven to be a massive challenge as evidenced by the multitude of implementation attacks over the years. Triggered off by research that showed how to utilise additional information via timing and power side channels\cite{DPA}, many different flavours of side channel attacks were discovered over the last decade. Many attacks use phyiscal information (such as low level execution timings or power consumption) to recover secret keys, but many other attacks use protocol level information (such as packet lengths, types of packets or protocol messages) to recover information about plaintexts, devices in the network, or the network itself. 
There exists a considerable body of work in the context of conventional, i.e. HTTPs over TCP/IP network, but the applicability of (some) of these attacks in the context of a typical IoT protocol stack is lacking. This is the gap that we would like to address with this work. 

This paper is structured as follows: after reviewing some relevant attack paths for HTTPs over TCP/IP in the following subsection, we provide a brief introduction to the necessary protocol and network features in \Cref{sec: IotProtocols}.  We discuss the impact of packet length leakage in \Cref{sec: PacketLen}, followed by an analysis of the response time leakage in \Cref{sec: ResponseTime}. We summarise our work in \Cref{sec: Conclusion}. 

\subsection{Related Work\label{subsec: RelatedWork}}
%Write about attacks that exploit packet length and response times in the context of TCP/IP. Reflect on web applications and how the same might happen if more interesting applications might be deployed that involve IoT objects.

%Literatures about Traffic Analysis...
Traffic Analysis is well studied in the context of encrypted Internet traffic, especially for web applications based on HTTPs and TCP/IP. The landmark study by Chen et al. \cite{WebSidechannel} discussed different side channel attacks against web applications and \cite{SuggestBox} studied the practicability of an attack specifically targeted Google and Bing search boxes. Later work by Mather and Oswald \cite{PinpointWeb} proposed the use of Mutual Information to pinpoint the potential leakage points in web traffic. For non-HTTPs applications, the papers \cite{AppleMsg}, \cite{Language} and \cite{VideoTraffic} described attacks against encrypted text, voice and video traffic respectively. Machine learning is widely used to analyse the traffic, and behaviours of different classifiers are studied by \cite{HClassifier} and \cite{Peekaboo}. Based on all these published works we can conclude that two features, the packet length and response time, are the most exploited ones among all attacks. Different countermeasures were studied by \cite{TrafficMorphing}, \cite{HTTPOS} and \cite{FTE}.

Reflecting on IoT applications, we stipulate that most of these attacks may still be applicable, as we intend to demonstrate in this paper. Considering the future vision that IoT devices could be indeed connected to the Internet with even more sensitive data flowing over different networks, the task of designing secure IoT applications becomes increasingly challenging.

%Literatures about 6LoWPAN security
With regard to the aspect of protocol design, the recent paper \cite{6LoWPANAtk} summarised some known flaws of 6LoWPAN, including its susceptibility to the Fragmentation Attack\cite{FragAtk}, Sinkhole Attack\cite{Sinkhole}, Hello Flood Attack\cite{HelloFlood}, Wormhole Attack\cite{Wormhole} and Blackhole Attack\cite{Blackhole}. In addition, \cite{802154SecIssues} reported certain problematic designs in 802.15.4 security\cite{802154}. However we do not discuss further these particular design flaws as they touch on a different aspect of the security issues in 6LoWPAN compared to what we address in this paper.
