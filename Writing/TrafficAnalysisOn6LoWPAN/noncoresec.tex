\section{802.15.4 Security in Contiki\label{noncoresec}}
%Noncoresec
802.15.4 security in implemented by Noncoresec\cite{noncoresec} in the latest Contiki  (release 3.0). One hard-coded network wise key is the only keying model it supports in the current release. Due to the absence of IPSec\cite{rfc4301}, noncoresec is effectively the only applicable approach for securing part of the network metadata, such as IP headers and TCP/UDP headers.

%Platform
%Most of our studies are done on traffic data generated by Cooja simulator\cite{Contiki}. However, TelosB\cite{TelosB} and CC2538\cite{CC2538} are also used to gather performance critical data, such as timings.

%Reset Attack and Anti-Replay
However, examining the code, we realised two issues exist in the current implementation:
\begin{itemize}
	\item \textbf{Nonce Reusing} \\
	\Cref{NoncoresecNonce} illustrates the construction of nonce in 802.15.4 security with AES-128 in CTR and CCM mode. According to the specification\cite{802154}, for a specific source, the Frame Counter is the only variable field among different packets. Noncoresec implemented Frame Counter as a static variable initialised to 0 upon each reboot. Such implementation eventually leads to the nonce being reused upon each reset of  the device, which in many cases could cause severe leakage to the plaintext.
	
	\begin{figure}[th!]
	\centering
	\adjustbox{max width = \textwidth}
	{
		\begin{tabular}{|c|c|c|c|c|}
			\hline 
			1 (bytes) & 8              & 4             & 1              & 2             \\ \hline
			Flags      & Source Address & Frame Counter & Security Level & Block Counter \\ \hline
		\end{tabular}
	}
	\caption{Nonce Construction in 802.15.4 Security}
	\label{NoncoresecNonce}
	\end{figure}
	
	\item \textbf{Packet Lost by Anti-replay} \\
	Even though the incompatibility between Anti-replay and network key has been pointed out by \cite{802154SecIssues}, the noncoresec implementation has mitigated the issue by using an extra data structure recording the last Frame Counter from each source address. However, this approach also induces a problem that packets sent by a rebooted device will be labelled as replays and thus dropped by the receiver.
\end{itemize}

Although nonce reusing could be mitigated by initialising Frame Counter to a random value on each reboot, or recording the latest value on a permanent media such as flash memory,  yet the 4 byte space can hardly be considered cryptographically secure. Therefore a full solution may not be achieved without updating the nonce construction specified by the standard\cite{802154}. 

The anti-replay issue is closely related to key management which remains an open question in the subject; thus would either be easily to solved without modifying the standard.  As a compromising solution, we simply recommend to disable the anti-replay feature and leave it to upper layer protocols such as TCP and CoAP.