\section{Characteristics of Sensor Network Traffics}

Due to the chracteristic ecology, traffic data in sensor networks need to be considered differently comparing to Internet.


First of all, various protocol suites are proposed for sensor networks for different application. The choice of different protocols results into different traffic meta data. For example, when using ContikiMAC\cite{ContikiMAC} (default in Contiki OS), the traffic meta data is similar to those we have for the Internet, whereas for TSCH\cite{rfc7554} there will be additional channel information. 
	
Secondly, unlike the dominant usage of TLS/SSL over Internet, encryption is considered on different layers for sensor networks, which also affects the information for an eavesdropping adversary. For instance the IPv6 header is visible for DTLS\cite{rfc6347} but is hidden when using 802.15.4 Security\cite{802154}.

Also that unreliable transmission is more preferable than reliable transmission for the constrained devices. This implies that, like UDP comparing to TCP, there might be no flow control which dynamically reassembles data into multiple packets. As a result, the actual size of data can be accurately observed from the packet size, which could provide more information to the adversary.

The absence of "random" data, such as ads, is another factor. The random data naturally randomises the traffic and hence served a countermeasure against the attacks. The leakage would be more easily observed without these noise.

%Real time applications; hence less noise in timing.
In aspect of packet timings, the constrained devices have extremely limited processor frequency and their system entropies are also relatively lower; therefore the correlation between response latency and code execution time would be stronger than those of Internet servers.

%More targets in addition to encrypted contents.
 
%Therefore more predictable packet features.

%\begin{itemize}
%	\item Different approaches: sophisticated techniques may not even required, e.g. machine learning, Mutual Information analysis. Leakage is easily observed.
%\end{itemize}

%Different scenarios have different leakages, e.g. the channels used in  TSCH could also be a leakage source.

%Existing security protocols does not protect these side channel information.
