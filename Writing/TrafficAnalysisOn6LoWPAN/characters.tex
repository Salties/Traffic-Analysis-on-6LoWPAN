\section{Observations to Sensor Network Traffic}

Sensor network traffic has several different points comparing to Internet traffic.

Various protocol stacks are proposed for different applications, some of them compete to each other. Different protocol generates different traffic meta data. For example, when using ContikiMAC\cite{ContikiMAC}\footnote{Default by Contiki OS.}, the types of traffic meta data is basically the same as of Internet; whereas for TSCH\cite{rfc7554} there will be additional channel information. 
	
Unlike the dominant usage of TLS/SSL over Internet, encryption could be applied on different layers of the protocol stack. Information the adversary has access to is therefore differs for each setting. For instance, the IPv6 header is visible under DTLS\cite{rfc6347} but is hidden under 802.15.4 Security\cite{802154}.

Unreliable transmission is preferable than reliable one due to the constrained resources. This implies that there might be no flow control which dynamically reassembles data into multiple packets, for example comparing UDP to TCP. Without the reassembling, number of packets is constant and the actual size of data can be accurately observed from the packet size, which could provide more information to the adversary. Although packet fragmentation is also provided by IPv6 but this feature is usually disabled for various reasons.

The absence of ``random" data, such as ads, is another major difference. The random data naturally randomises the traffic and hence complicates traffic analysis attacks. The leakage would be more easily observed without these noise.

%Real time applications; hence less noise in timing.
Timing measurements are also more accurate in sensor networks. Firstly, sensors are physically more close to each other and uses RF to communicate. This enables the adversary to remove the RTT noises by directly measure the server side response time of each packet. Secondly, the processor frequencies are also lower for the constrained devices and hence gives a better resolution on code execution time.

%More targets in addition to encrypted contents.
Finally, information other than encrypted contents could also be interested targets in case of sensor network. For example, network structure may help the adversary to launch DoS attacks more efficiently, and hardware information might be exploited to launch attacks against devices with specific vulnerabilities.

%Therefore more predictable packet features.
In summary, many natural of sensor network has only made it even more vulnerable against traffic analysis attacks. As a matter of fact, later in this paper, we demonstrate some examples of information leakage that can be directly observed through a sniffer, without utilising many sophisticated techniques developed for Internet such as machine learning classifiers.
