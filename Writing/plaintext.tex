\section{Leakage on Encrypted Contents}

%A primary target of traffic analysis attacks is the plaintext in the encrypted contents. These attacks normally requires the adversary to have the pre-knowledge of a finite set of the potential plaintext; therefore each attack is strongly dependent to a specific application.

Many traffic analysis attacks rely heavily on pre-knowledge of communication contents. However, as of this study, it would be inappropriate to make further assumptions about the data being transmitted due to the variety of sensor network applications. Without loss of generality, in this section, we study   two cases of relatively mature standards in the community, namely ICMPv6 and CoAP, respectively.

\paragraph{Types of ICMP Messages}

Internet Control Message Protocol(ICMP)\cite{rfc4443} performs the management tasks in the network, such as link establishment and routing information exchange. Contiki OS supports the following ICMP messages:

\begin{itemize}
	\item \textbf{DAG Information Object (DIO)} \\
	DIO contains the 6LoWPAN global information. It could be periodically broadcasted for network maintenance, or unicasted to a new joining node as a reply to DIS (see below).
	\item \textbf{DAG Information Solicitation (DIS)} \\
	DIS is sent by a newly started node to probe any existing 6LoWPANs. A DIO would be replied if the DIS is received by any neighbour nodes.
	\item \textbf{Destination Advertisement Object (DAO)} \\
	DAO is sent by a child node to its precedents\footnote{The 6LoWPAN DODAG topology is defined in \cite{rfc6550}.} to propagate its routing information.
	\item \textbf{Neighbour Solicitation (NS) and Neighbour Advertisement (NA)} \\
	NS and NA are the ARP replacement in IPv6, where NS queries a translation and NA answers one. In addition, they are also used for local link validity check.
	\item \textbf{Echo Request and Echo Response (PING)} \\
	Echo Request and Echo Response are well known as the PING packets. They are mostly used for diagnostic purposes, such as connectivity test or Round Trip Time (RTT) estimation. Echo Request may contain arbitrary user defined data and Echo Response simply echoes its corresponding request.
\end{itemize}

Generally, ICMP messages can be protected by either using the secure ICMP messages, or relying on lower layer encryption. Since Contiki OS does not have the former implemented, 802.15.4 Security became the only option currently.

Although leakage in ICMP messages does not directly lead to any breach of application data, it would still be harmful by providing the adversary with information including network topology. These informations could then be exploited to, say, identify critical targets of DoS attacks. 

We simulated a 6LoWPAN network with 802.15.4 Security enabled\footnote{Security level is set to the highest 0x07.}. The nodes are configured to also generate random UDP packets. 

Despite the ICMP messages are encrypted by 802.15.4 Security, our experiment shows that several ICMP messages can still be identified by the packet size and MAC destination. 

\Cref{IcmpPacketFeature} summarises the packet features. $x$ to denotes the size of user defined data in bytes.

\begin{table}
	\center
	{
	\begin{tabular}{|c|c|c|}
		\hline
		       & Packet Size (bytes) & Type of MAC Destination \\ \hline
		DIS    & 85                  & broadcast                       \\ \hline
		DIO  & 118/123                 & broadcast/unicast                       \\ \hline
		DAO    & 97                  & unicast                      \\ \hline
		NS & 87                  & broadcast/unicast                       \\ \hline
		NA     & 87                  & unicast                      \\ \hline
		PING   & $101+x$               & unicast                      \\ \hline
		UDP Multicast   & $85+x$                  & broadcast               \\ \hline
		UDP Unicast   & $107+x$                  & unicast                       \\ \hline
	\end{tabular}
}

	\caption{6LoWPAN Packet Features\label{IcmpPacketFeature}}
\end{table}

%First of all, a DIS has the unique smallest packet size of 85 bytes, indicating it can be easily identified.

Among the unicast packets, since PING and UDP have at least 101 and 108 bytes\footnote{PING can be sent without user defined data and UDP packets requires at least 1 byte.}, DAO can therefore uniquely identified as unicast packets of $97$ bytes. 

For the same reason NA and unicast NS can also be distinguished from other packets by filtering packets of $87$ bytes. Considering NA is sent as a response to NS according to the protocol, one can always identify the first being NS and second NA. 

Similarly, unicast DIO can be identified as the 123 bytes packet followed by DIS, where the later has a unique 85 bytes size. However, there is a potential of false positive induced by carefully crafted PING or UDP packets.

PING could be recognised by its pair-wised appearance, as the response would have nearly the same meta data as the original request, except the exchanged source and destination.

For broadcast packets, DIS can be easily identified by its unique 85 bytes packet size. Others like broadcast NS can be identified by the followed characteristic NA response; and packets of 118 bytes those are periodically broadcasted are likely to be DIOs.

In summary, among all the packets, DAO, NA, NS, DIS can be identified with certain. DIO and PING cannot be certainly identified by they both have significant characters. Notice that the above contained all ICMPv6 messages supported; therefore UDP packets can be reversely filtered, although in some cases mixed with DIO and PING.

\paragraph{Sensor Readings with CoAP}

CoAP\cite{rfc7252} is a protocol designed for constraied devices that provides an universal interface for accessing resources. CoAPs is the secure version which stands for CoAP with DTLS.

Due to the different physical characters of sensors, there could be a variance of time when reading the measurements. The idea is to investigate whether such variance can be observed through the packet response latency.

We implemented the experiment on CC2538, using all three sensors from ``cc2538-demo", namely Vdd, temperature and Ambient Light Sensor (ALS). We used CoAP from the ``er-rest-example" in the Contiki OS source code, as there is no CoAPs implementation available. 

Although DTLS processing would definitely have an impact on the response latency, we argue that such impact would be independent to the sensors being accessed; hence similar result would hold even in case of CoAPs.

We have carefully crafted other factors, including URIs, data representation and code flow, to be uniform for all three sensors in order to guarantee a controlled environment.

\begin{table}
	\center
	\begin{tabular}{|c|c|c|}
	\hline
	& Average (ms) & Range(ms) \\ 
	\hline
	Vdd & 9.622 & [9.388, 10.318] \\ 
	\hline
	Temperature & 9.835 & [9.525, 10.318] \\ 
	\hline
	ALS & 11.651 & [11.338, 12.031] \\
	\hline
\end{tabular} 
	\caption{CoAP Response Latency for Sensor Readings on CC2538\label{CoapTiming}}
\end{table}

\Cref{CoapTiming} summarises the result. It is shown that ALS takes about $2$ms longer and hence can be easily distinguished. Vdd and temperature might be difficult to distinguish by response latency as they have similar results.

%To summarise, the above examples confirmed the applicability of classic traffic analysis attacks in case sensor networks. The leakages are significant enough to be observed without utilising any sophisticated analytical tools.
